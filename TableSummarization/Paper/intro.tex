%!TEX root = TableSummarization.tex


\section{Introduction}
Analysts often use OLAP (Online Analytical Processing) operations
such as drill-down (and roll-up)~\cite{export:69578} to explore
relational databases. 
These operations are very useful for analytics and data exploration and have stood the test of time;
all commercial OLAP systems 
in existence support these operations. (Recent reports estimate the size of the OLAP market to be \$10+ Billion~\cite{gartner}.)


However, there are cases where drill-down is ineffective; 
for example, when the number of distinct values
in a column is large, vanilla drill-down 
could easily overwhelm analysts by presenting them with too many 
results (i.e., aggregates). 
Further, drill-down only allows us to instantiate values
 one column at a time, instead of allowing simultaneous drill-downs
on multiple columns---this simultaneous drill-down on multiple columns 
could once again suffer from the problem
of having too many results, stemming from many distinct combinations of column values.

In this paper, we present a new interaction operator 
that is an extension to a traditional 
drill-down operator, aimed at providing {\em complementary}
functionality to drill-down in cases where drill-down is
ineffective. We call our operator {\em smart drill-down}.
At a high level, smart drill-down lets analysts zoom into
the more ``interesting'' parts of a table or a database,
with fewer operations, and without having to examine as much
data as traditional drill-down.
Note that our goal is {\em not} to replace traditional 
drill-down functionality, which we believe is fundamental;
instead, our goal is to provide auxiliary functionality 
which the analyst is free to use whenever they find 
traditional drill-downs ineffective.

In addition to presenting this new operator called smart drill-down, in this paper, 
we also present novel sampling techniques to compute the results for this
operator {\em in an interactive fashion} on increasingly larger databases. 
Unlike the traditional OLAP setting, these computations 
require no pre-materialization, and can be implemented 
within or on top of any relational database system.



% drill-down (and roll up)~\cite{export:69578} are very useful tools for exploring
% relational tables.
% In this paper we study an extension of drill-down
% that lets users zoom into the more ``interesting'' parts of a table,
% with fewer operations and without having to examine as much data
% as traditional drill-down.
% We call our approach {\em smart drill-down};
% it does not replace traditional drill-down but we will argue is a very
% useful complement.
The best way to explain smart drill-down is through a simple example.

\begin{example}\label{ex:introexample}
Consider a table with columns `Department Store', `Product', `Region'
and `Sales'. Suppose an analyst queries for tuples
where Sales were higher than some threshold, in order
to find the best selling products.
If the resulting table has many tuples,
the analyst can use traditional drill-down to explore it.
For instance, the system may initially tell the analyst there are
6000 tuples in the answer, represented by the tuple ($\star$, $\star$, $\star$, $6000$, $0$),
as shown in Table~\ref{table:introexample0}.
The $\star$ character is a wildcard that matches any value in the database.
The Count attribute can be replaced by a Sum aggregate over some measure column,
e.g., the total sales.
The right-most Weight attribute is the number of non-$\star$ attributes; 
its significance will be discussed shortly.
If the analyst drills down on the Store attribute (first $\star$),
then the operator displays all tuples of the form ($X$, $\star$, $\star$, $C$, $1$),
where $X$ is a Store in the answer table, and $C$
is the number of tuples for $X$ (or the aggregate sales for $X$).

Instead, when the analyst uses smart drill-down on Table~\ref{table:introexample0},
he obtains Table~\ref{table:introexample}.
The ($\star$, $\star$, $\star$, $6000$) tuple is expanded into $3$ tuples
that display noteworthy or interesting drill-downs.
The number $3$ is a user specified parameter, which we call $k$.

For example, the tuple (Target, bicycles, $\star$, $200$, $2$)
says that there are $200$ tuples (out of the 6000) with
Target as the first column value and bicycle as the second.
This fact tells the analyst that Target is selling a lot of bicycles.
The next tuple tells the analyst that comforters are selling well in
the MA-3 region, across multiple stores. The last tuple
states that Walmart is doing well in general over multiple products and regions.
We call each tuple in Table~\ref{table:introexample} a {\em rule}
to distinguish it from the tuples in the original table that is being explored.
Each rule summarizes the set of tuples that are described by it.
Again, instead of Count, the operator can display a Sum aggregate, such as
the total Sales.

\begin{table}
\scriptsize
\centering
\begin{tabular}{| l | l | l | l | l |}
\hline Store & Product & Region & Count & Weight \\
\hline
$\star$ & $\star$ & $\star$ & $6000$ & $0$ \\ \hline
\end{tabular}
\vspace{-10pt}
\caption{Initial summary}\label{table:introexample0}
\end{table}

\begin{table}
\scriptsize
\centering
\begin{tabular}{| l | l | l | l | l |}
\hline Store & Product & Region & Count & Weight \\
\hline
$\star$ & $\star$ & $\star$ & $6000$ & $0$ \\ \hline
$\triangleright$ Target & bicycles & $\star$ & $200$ & $2$ \\ \hline
$\triangleright$ $\star$ & comforters & MA-3 & $600$ & $2$ \\ \hline
$\triangleright$ Walmart & $\star$ & $\star$ & $1000$ & $1$ \\ \hline
\end{tabular}
\vspace{-10pt}
\caption{Result after first smart drill-down}\label{table:introexample}
\vspace{-10pt}
\end{table}

\begin{table}
\scriptsize
\centering
\vspace{-5pt}
\begin{tabular}{| l | l | l | l | l |}
\hline Store & Product & Region & Count & Weight \\
\hline
$\star$ & $\star$ & $\star$ & $6000$ & $0$ \\  \cline{1-5}
$\triangleright$ Target & bicycles & $\star$ & $200$ & $2$ \\ \cline{1-5}
$\triangleright$ $\star$ & comforters & MA-3 & $600$ & $2$ \\ \cline{1-5}
$\triangleright$ Walmart & $\star$ & $\star$ & $1000$ & $1$ \\ \cline{2-5}
$\triangleright$ $\triangleright$ Walmart & cookies & $\star$ & $200$ & $2$ \\ \cline{2-5}
$\triangleright$ $\triangleright$ Walmart & $\star$ & CA-1 & $150$ & $2$ \\ \cline{2-5}
$\triangleright$ $\triangleright$ Walmart & $\star$ & WA-5 & $130$ & $2$ \\ \hline
\end{tabular}
\vspace{-10pt}
\caption{Result after second smart drill-down} \label{table:introexample2}
\vspace{-10pt}
\end{table}

Say that after seeing the results of Table~\ref{table:introexample},
the analyst wishes to dig deeper into the Walmart tuples
represented by the last rule.
For instance, the analyst may want to know
which states Walmart has more sales in, or which products they sell
the most. In this case, the analyst clicks on the Walmart rule,
obtaining the expanded summary in Table~\ref{table:introexample2}.
The three new rules in this table provide additional information
about the $1000$ Walmart tuples.
In particular, one of the new rules shows that
Walmart sells a lot of cookies and the others show it sells a lot of products in
the regions CA-1 and WA-5.

When the analyst clicks on a rule $r$, smart drill-down
expands $r$ into $k$ sub-rules that as a set are deemed to be ``interesting.''
(We discuss other smart drill-down operations in Section~\ref{sec:interface}.)
There are three factors that make a rule set interesting.
One is if contains rules with high Count (or total sales) fields,
since the larger the count, the more tuples are summarized.
A second factor is if the rules have high weight (number of non-$\star$ attributes).
For instance, the rule (Walmart, cookies, AK-1, $200$, $3$)
seems more interesting than (Walmart, cookies, $*$, $200$, $2$)
since the former rule tells us the high sales are concentrated in a single region.
A third desirability factor is diversity:
For example, if we already have the rule (Walmart, $\star$, $\star$, $1000$, $1$)
in our set, we would rather have the rule (Target, bicycles, $\star$, $200$, $2$)
than (Walmart, bicycles, $\star$, $200$, $2$) since the former rule
describes tuples that are not described by the first rule.

In this paper we describe how to combine or blend these three factors
in order to obtain a single desirability score for a set of rules.
Our score function can actually be tuned by the analyst
(by specifying how weights are computed),
providing significant flexibility in what is considered a good set of rules.
We also present an efficient optimization procedure to maximize score, invoked
by smart drill-down to select the set of $k$ rules to display.

\end{example}

Compared to traditional drill-down, our smart drill-down has two important advantages:
\squishlist
\item
Smart drill-down limits the information displayed
to the most interesting $k$ facts (rules).
With traditional drill-down, a column is expanded and {\em all}
attribute values are displayed in arbitrary order.
In our example, if we drill-down on say the store attribute,
we would see all stores listed, which may be a very large number.
\item
Smart drill-down explores several attributes to open up together,
and automatically selects combinations that are interesting.
For example, in Table~\ref{table:introexample},
the rule (Target, bicycles, $\star$, $200$, $2$)
is obtained after a single drill-down;
with a traditional approach, the analyst would first have to drill-down on
Store, examine the results, drill-down on Product,
look through all the displayed rules and then find the interesting rule
(Target, bicycles, $\star$, $200$, $2$).
\squishend

Incidentally, note that in the example we only described one type of smart drill-down,
where the analyst selects a {\em rule} to drill-down on
(e.g., the Walmart rule going from Table~\ref{table:introexample} to
Table~\ref{table:introexample2}.
In Section~\ref{sec:interface} we describe another option
where the analyst clicks on a $\star$ in a column to obtain
rules that have non-$\star$ values in that column.

Our work on smart drill-down is related
to table summarization and anomaly
detection~\cite{Sarawagi:2001:UMA:767141.767148,
Sarawagi00user-adaptiveexploration,
Sarawagi98discovery-drivenexploration,
DBLP:journals/pvldb/GebalyAGKS14}.
These papers mostly focus on
giving the most ``surprising'' information to the user, i.e., information
that would minimize the Kullback-Liebler(KL) divergence between the
resulting maximum entropy distribution and the actual value distribution. For instance, if a certain set of
values occur together in an unexpectedly small number of tuples, that
set of values may be displayed to the user. In contrast, our algorithm
focuses on rules with high counts, covering as
much of the table as possible.
Furthermore, our summarization is couched in
an interactive environment, where the analyst
directs the drill-down and can tailor the optimization criteria.
Nevertheless, one can envision extending traditional
and smart drill-down to provide an additional option of anomaly detection. We discuss related work in detail in  Section~\ref{sec:related}.

To reiterate, our chief contribution in this paper is the {\em smart drill-down} interaction operator,
an extension of traditional drill-down, aimed at allowing analysts to zoom into the
more ``interesting'' parts of a dataset. In addition to this operator,
we develop techniques to support this operator on increasingly larger datasets:
\squishlist
\item {\em Basic Interaction:} We demonstrate that finding the optimal list of rules is {\sc NP-Hard},
and we develop an algorithm to find the approximately optimal list of rules to display when the user performs a smart drill-down operation. b
\item {\em Dynamic Sample Maintenance:} To improve response time on large tables, we formalize the problem of dynamically
maintaining samples in memory to support smart drill-down. 
We show that optimal identification of samples is once again {\sc NP-Hard}, \agp{??} and 
we develop an approximate scheme for dynamically maintaining and using multiple samples of the table in memory.
% to improve response time on large tables.
\squishend

\stitle{Overview of paper:} 
\squishlist 

\item In Section~\ref{sec:formal}, we formally define smart drill-down. After that, we describe different schemes for weighting rules, and our interactive user interface.

\item In Section~\ref{sec:algorithms}, we present our algorithms for
finding optimal sets of rules, as well as our dynamic sampling schemes
for dealing with large tables.

\item Based on our implemented smart drill-down,
in Section~\ref{sec:experiments} we experimentally evaluate
performance on real datasets,
and show additional examples of smart drill-down in action.

\item We describe related work in Section~\ref{sec:related}, and conclude in Section~\ref{sec:conclusion}.
\squishend 
